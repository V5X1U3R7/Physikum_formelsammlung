% Formel für Lichtgeschwindigkeit
\begin{equation}
    c = \lambda \times \nu \; \mathrm{\left[\frac{m}{s}\right]}
\end{equation}

% Wellenlänge und Frequenz ist umgekehrt-proporional
\begin{equation}
    \lambda \times \nu = \mathrm{konstant}
\end{equation}

% Wirkungsquantum
\begin{equation}
    h = \frac{E}{\nu}
\end{equation}

% Mathe-Abstände; v.l.n.r größer werdend
\begin{align}
    a^2 \! + \! b^2 &= c^2 \\
    a^2 \, + \, b^2 &= c^2 \\
    a^2 \: + \: b^2 &= c^2 \\
    a^2 \; + \; b^2 &= c^2 \\
    a^2 \ + \ b^2 &= c^2 \\
    a^2 \quad + \quad b^2 &= c^2 \\
    a^2 \qquad + \qquad b^2 &= c^2
\end{align}