\documentclass{article}

\usepackage{amsmath}

\begin{document}

\vfill

\begin{center}
    {\Large Formelsammlung fürs Physikum}\footnote{Das Dokument dient als Veranschaulichung der Formeln und ist lediglich für interne Zwecke gedacht.}

    \vspace{1cm}
    
    Geschrieben von Niclas Thiebach
\end{center}

\vfill

\newpage

\section{Grundlegendes}

\subsection{Abstände im Mathemodus}
\begin{align*}
    a^2 \! + \! b^2 &= c^2 \\
    a^2 \, + \, b^2 &= c^2 \\
    a^2 \: + \: b^2 &= c^2 \\
    a^2 \; + \; b^2 &= c^2 \\
    a^2 \ + \ b^2 &= c^2 \\
    a^2 \quad + \quad b^2 &= c^2 \\
    a^2 \qquad + \qquad b^2 &= c^2
\end{align*}


\section{Formeln}

\subsection{Formel für die Lichtgeschwindigkeit}
\begin{equation}
    c = \lambda \times \nu \; \mathrm{\left[\frac{m}{s}\right]}
\end{equation}

\subsection{Proportionalität zwischen Wellenlänge und Frequenz}
\begin{equation}
    \lambda \times \nu = \mathrm{konstant}
\end{equation}

\subsection{Formel für das Wirkungsquantum}
\begin{equation}
    h = \frac{E}{\nu}
\end{equation}

\end{document}