\documentclass{article}

\usepackage{amsmath}
\usepackage{tasks}

\begin{document}

\vfill

\begin{center}
    {\Large Formelsammlung fürs Physikum}\footnote{Das Dokument dient als Veranschaulichung der Formeln und ist lediglich für interne Zwecke gedacht.}

    \vspace{1cm}
    
    Geschrieben von Niclas Thiebach
\end{center}

\vfill

\newpage

\section{Grundlegendes}

\subsection{Abstände im Mathemodus}
\begin{align*}
    a^2 \! + \! b^2 &= c^2 \\
    a^2 \, + \, b^2 &= c^2 \\
    a^2 \: + \: b^2 &= c^2 \\
    a^2 \; + \; b^2 &= c^2 \\
    a^2 \ + \ b^2 &= c^2 \\
    a^2 \quad + \quad b^2 &= c^2 \\
    a^2 \qquad + \qquad b^2 &= c^2
\end{align*}


\section{Formeln}

\subsection{Formel für die Lichtgeschwindigkeit}
\begin{equation}
    c = \lambda \times \nu \; \mathrm{\left[\frac{m}{s}\right]}
\end{equation}

\subsection{Proportionalität zwischen Wellenlänge und Frequenz}
\begin{equation}
    \lambda \times \nu = \mathrm{konstant}
\end{equation}

\subsection{Formel für das Wirkungsquantum}
\begin{equation}
    h = \frac{E}{\nu}
\end{equation}


\section{Griechische Buchstaben}

\begin{center}
\begin{tabular}{ll|ll}
    \textbf{Symbol} & \textbf{Name} & \textbf{Symbol} & \textbf{Name} \\
    \hline
    $\alpha$ & Alpha & $\nu$ & Nu \\
    $\beta$ & Beta & $\xi,\;\Xi$ & Xi \\
    $\gamma,\;\Gamma$ & Gamma & $\pi,\;\Pi$ & Pi \\
    $\delta,\;\Delta$ & Delta & $\rho$ & Rho \\
    $\epsilon$ & Epsilon & $\sigma,\;\Sigma$ & Sigma \\
    $\zeta$ & Zeta & $\tau$ & Tau \\
    $\eta$ & Eta & $\upsilon,\;\Upsilon$ & Upsilon \\
    $\theta,\;\Theta$ & Theta & $\phi,\;\Phi$ & Phi \\
    $\iota$ & Iota & $\chi$ & Chi \\
    $\kappa$ & Kappa & $\psi,\;\Psi$ & Psi \\
    $\lambda,\;\Lambda$ & Lambda & $\omega,\;\Omega$ & Omega \\
    $\mu$ & Mu & $\varepsilon$ & Epsilon (Variante) \\
    $\vartheta$ & Theta (Variante) & $\varphi$ & Phi (Variante) \\
    $\varpi$ & Pi (Variante) & $\varrho$ & Rho (Variante) \\
    $\varsigma$ & Sigma (Variante) & & \\
\end{tabular}
\end{center}

\section{Biochemie FS4}

\subsection{Verdünnugnsrechnungen}

\begin{align}
    \mathrm{Extinktion} &= \mathrm{Konzentration} \times \mathrm{Steigung} + \mathrm{Achsenabschnitt}\\
    \mathrm{Konzentration} &= \frac{\mathrm{Extinktion} - \mathrm{Achsenabschnitt}}{\mathrm{Steigung}}
\end{align}

\end{document}