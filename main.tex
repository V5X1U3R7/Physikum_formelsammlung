\documentclass{article}

\usepackage{amsmath}

\begin{document}

\vfill

\begin{center}
    {\Large Formelsammlung fürs Physikum}\footnote{Das Dokument dient als Veranschaulichung der Formeln und ist lediglich für interne Zwecke gedacht.}

    \vspace{1cm}
    
    Geschrieben von Niclas Thiebach
\end{center}

\vfill

\newpage

\section{Grundlegendes}

\subsection{Abstände im Mathemodus}
\begin{align*}
    a^2 \! + \! b^2 &= c^2 \\
    a^2 \, + \, b^2 &= c^2 \\
    a^2 \: + \: b^2 &= c^2 \\
    a^2 \; + \; b^2 &= c^2 \\
    a^2 \ + \ b^2 &= c^2 \\
    a^2 \quad + \quad b^2 &= c^2 \\
    a^2 \qquad + \qquad b^2 &= c^2
\end{align*}


\section{Formeln}

\subsection{Formel für die Lichtgeschwindigkeit}
\begin{equation}
    c = \lambda \times \nu \; \mathrm{\left[\frac{m}{s}\right]}
\end{equation}

\subsection{Proportionalität zwischen Wellenlänge und Frequenz}
\begin{equation}
    \lambda \times \nu = \mathrm{konstant}
\end{equation}

\subsection{Formel für das Wirkungsquantum}
\begin{equation}
    h = \frac{E}{\nu}
\end{equation}


\section{Griechische Buchstaben}

\begin{itemize}
    \item[$\alpha$] Alpha
    \item[$\beta$] Beta
    \item[$\gamma,\;\Gamma$] Gamma
    \item[$\delta,\;\Delta$] Delta
    \item[$\epsilon$] Epsilon
    \item[$\zeta$] Zeta
    \item[$\eta$] Eta
    \item[$\theta,\;\Theta$] Theta
    \item[$\iota$] Iota
    \item[$\kappa$] Kappa
    \item[$\lambda,\;\Lambda$] Lambda
    \item[$\mu$] Mu
    \item[$\nu$] Nu
    \item[$\xi,\;\Xi$] Xi
    \item[$\pi,\;\Pi$] Pi
    \item[$\rho$] Rho
    \item[$\sigma,\;\Sigma$] Sigma
    \item[$\tau$] Tau
    \item[$\upsilon,\;\Upsilon$] Upsilon
    \item[$\phi,\;\Phi$] Phi
    \item[$\chi$] Chi
    \item[$\psi,\;\Psi$] Psi
    \item[$\omega,\;\Omega$] Omega
    \item[$\varepsilon$] Epsilon (variante)
    \item[$\vartheta$] Theta (variante)
    \item[$\varphi$] Phi (variante)
    \item[$\varpi$] Pi (variante)
    \item[$\varrho$] Rho (variante)
    \item[$\varsigma$] Sigma (variante)
\end{itemize}

\end{document}